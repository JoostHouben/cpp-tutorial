\documentclass[12pt,a4paper]{article}
\usepackage[a4paper,margin=2.5cm]{geometry}
\usepackage[utf8]{inputenc}
\usepackage[T1]{fontenc}
\usepackage[dutch]{babel}
\usepackage{amssymb, amsmath}
\usepackage{fancyhdr}
\usepackage{footnote}
\usepackage{hyperref}
\usepackage[justification=centering]{caption}
\usepackage{enumerate}
\usepackage{booktabs}
\usepackage{graphicx}
\usepackage{wrapfig}
\usepackage{bm}
\usepackage{siunitx}
\usepackage[usenames,dvipsnames]{color}
\pagestyle{fancy}
\setlength{\headheight}{15pt}
\setlength{\parindent}{0pt}

\renewcommand{\thesection}		{\arabic{section}}
\renewcommand{\thesubsection}	{\arabic{subsection}}

\usepackage{listings}
\lstset{
    language=C++,
    tabsize=4,
    frame=lines,
    numbers=left,
    numberstyle=\color{black}\tiny,
    numbersep=8pt,
    breaklines=true,
    showstringspaces=false,
    basicstyle=\color{black}\footnotesize\ttfamily,
	identifierstyle=\color{black},
	keywordstyle=\bfseries\color{blue}\bfseries,
	commentstyle=\color{OliveGreen},
	stringstyle =\color{red},
}

% \code to include a file
% \icode to create inline code
\newcommand{\code}{\lstinputlisting}
\newcommand{\icode}{\lstinline}
\newcommand{\mono}{\texttt}

\fancyhead[R]{}
\fancyfoot[C]{\thepage}

\title{Inleiding programmeren in C++\\
Arrays en while loops}
\author{
FysiCie\\
i.s.m.\\
Ragnar Groot Koerkamp
}
\date{11 juni 2015}
 
%\renewcommand{\contentsname}{Inhoud}

\newcommand{\cpp}{\mono{C++ }}
 
\begin{document}
 
\maketitle

\tableofcontents

\clearpage

\subsection{Arrays [optioneel]}

\code{code/array.cpp}

\subsection{While [optioneel]}
Tot slot hebben we nog de \icode{while}-loop. Deze lijkt een beetje op de \icode{for}-loop, maar is toch anders. Bij een \icode{for}-loop weten we meestal van tevoren al hoe vaak we de code willen herhalen. Als we dat niet precies weten, is een \icode{while}-loop handig. We herhalen de code nu net zo lang tot \icode{conditie} onwaar wordt in
\begin{lstlisting}
while(conditie){
	code die herhaald wordt;
}
\end{lstlisting}
In de code hieronder zie je een voorbeeld. We beginnen hier met $n=0$, en vervolgens schrijven we steeds $n^2$ op, net zo lang tot $n^2$ groter wordt dan $1000$.

\code{code/while.cpp}

\end{document}
